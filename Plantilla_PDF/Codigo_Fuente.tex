%% LyX 2.3.6.1 created this file.  For more info, see http://www.lyx.org/.
%% Do not edit unless you really know what you are doing.
\documentclass[12pt,twoside,spanish]{Plantilla}
\usepackage[T1]{fontenc}
\usepackage[latin9]{inputenc}
\usepackage[landscape,letterpaper]{geometry}
\geometry{verbose,tmargin=5cm,bmargin=3cm,lmargin=3cm,rmargin=3cm}
\setcounter{secnumdepth}{5}
\setcounter{tocdepth}{3}
\usepackage{babel}
\addto\shorthandsspanish{\spanishdeactivate{~<>}}

\usepackage{pifont}
\usepackage{float}
\usepackage[unicode=true,pdfusetitle,
 bookmarks=true,bookmarksnumbered=false,bookmarksopen=false,
 breaklinks=false,pdfborder={0 0 0},pdfborderstyle={},backref=false,colorlinks=false]
 {hyperref}

\makeatletter

%%%%%%%%%%%%%%%%%%%%%%%%%%%%%% LyX specific LaTeX commands.
%% Because html converters don't know tabularnewline
\providecommand{\tabularnewline}{\\}

%%%%%%%%%%%%%%%%%%%%%%%%%%%%%% User specified LaTeX commands.
\usepackage{Plantilla}

\AtBeginDocument{
  \def\labelitemi{\ding{109}}
  \def\labelitemii{\ding{113}}
  \def\labelitemiii{\Pisymbol{psy}{68}}
  \def\labelitemiv{\ding{226}}
}

\makeatother

\begin{document}

\Titulo{Nombre del Software}

\Subtitulo{C�digo Fuente}

\Autores{%
\begin{tabular}{c}
Autor 1\tabularnewline
Autor 2\tabularnewline
Autor 3\tabularnewline
\end{tabular}}

\AutoresCorto{Autor 1, Autor 2, Autor 3}

\Institucion{Nombre de la Universidad}

\Ciudad{Nombre de la Ciudad}

\Fecha{A�o de Publicaci�n}

\Logo{Imagenes/Logo.png}

\Version{1.22.12.2}

\GenerarPortada{}

\tableofcontents{}

\TipoCapitulo{Cap�tulo}

\chapter{C�digo Fuente}

Lorem ipsum dolor sit amet, consectetur adipiscing elit. In finibus
mi sit amet lectus varius, mattis ullamcorper felis dapibus. In a
est fermentum, tincidunt tellus quis, condimentum odio. Phasellus
facilisis egestas sem, at mattis quam sodales vel. Proin pulvinar
sodales faucibus. Morbi semper, erat vulputate vulputate posuere,
metus turpis cursus tortor, a commodo erat turpis et lectus. Vivamus
quis turpis at nulla placerat consequat ut sed nulla. In arcu turpis,
porta ac vestibulum vel, tristique in erat. Quisque eu turpis consectetur,
accumsan nulla ut, luctus nunc. Morbi porta varius ultricies.

\section{Facultades.js}

Descripci�n del C�digo

\lstset{style=codigoJavascript}

\begin{lstlisting}
const { pool, MENSAJE_ERROR_SERVIDOR } = require("../config/config");
const Utilidades = require("./utilidades.model");

class Facultad {

    async postFacultad(datos) {
       return new Promise((resolve, reject) => {
    
          let idfacultad = new Utilidades().generarID();
          let {facultad} = datos;
          let query = `INSERT INTO public.facultades (idfacultad, facultad) VALUES ($1, $2)`;
          pool.query(query, [idfacultad, facultad], (error, results) => {
             if (error) {
                reject({
                   Response: MENSAJE_ERROR_SERVIDOR + error,
                   Status: false
                });
                return;
             }
    
             resolve({
                Response: 'La facultad se cre� correctamente',
                Status: true
             });
          });
       });
    }

    async putFacultad(datos) {
       return new Promise((resolve, reject) => {
    
          let {idfacultad, facultad} = datos;
          let query = `UPDATE public.facultades SET facultad=$1 WHERE idfacultad='${idfacultad}'`;
          pool.query(query, [facultad], (error, results) => {
             if (error) {
                reject({
                   Response: MENSAJE_ERROR_SERVIDOR + error,
                   Status: false
                });
                return;
             }
    
             resolve({
                Response: 'La facultad se actualiz� correctamente',
                Status: true
             });
          });
       });
    }   

    async getFacultadById(idfacultad) {
       return new Promise((resolve, reject) => {
          let query = `SELECT * FROM public.facultades WHERE idfacultad='${idfacultad}'`;
          pool.query(query, (error, results) => {
             if (error) {
                reject({
                   Response: MENSAJE_ERROR_SERVIDOR + error,
                   Status: false
                });
                return;
             }
    
             resolve({
                Response: 'La facultad se obtuvo correctamente',
                Status: true,
                Object: results.rows[0]
             });
          });
       });
    }

    async getFacultades() {
       return new Promise((resolve, reject) => {
          let query = `SELECT * FROM public.facultades ORDER BY facultad`;
          pool.query(query, (error, results) => {
             if (error) {
                reject({
                   Response: MENSAJE_ERROR_SERVIDOR + error,
                   Status: false
                });
                return;
             }
    
             resolve({
                Response: 'Las facultades se obtuvieron correctamente',
                Status: true,
                Results: results.rows
             });
          });
       });
    }

    async deleteFacultad(idfacultad) {
       return new Promise((resolve, reject) => {
          let query = `DELETE FROM public.facultades WHERE idfacultad='${idfacultad}'`;
          pool.query(query, (error, results) => {
             if (error) {
                reject({
                   Response: MENSAJE_ERROR_SERVIDOR + error,
                   Status: false
                });
                return;
             }
    
             resolve({
                Response: 'La facultad se elimin� correctamente',
                Status: true
             });
          });
       });
    }
}

module.exports = Facultad;
\end{lstlisting}
\end{document}
